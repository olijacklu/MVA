\documentclass{article}
\usepackage{graphicx}
\usepackage{hyperref}
\usepackage{algorithm,algorithmic,caption} % lecture 5
\usepackage{amsmath}
\usepackage[T1]{fontenc}
\usepackage{amsfonts}
\usepackage{enumitem}
\usepackage{natbib}
\usepackage{listings}
\setenumerate[1]{label=\thesection.\arabic*.}
\usepackage{datetime}

\newcommand{\wt}{\widetilde}
\newcommand{\wh}{\widehat}


\newcommand{\ba}{\bold{a}}
\newcommand{\bb}{\bold{b}}
\newcommand{\bc}{\bold{c}}
\newcommand{\bd}{\bold{d}}
\newcommand{\be}{\bold{e}}
\newcommand{\bof}{\bold{f}}
\newcommand{\bg}{\bold{g}}
\newcommand{\bh}{\bold{h}}
\newcommand{\bi}{\bold{i}}
\newcommand{\bj}{\bold{j}}
\newcommand{\bk}{\bold{k}}
\newcommand{\bl}{\bold{l}}
\newcommand{\bm}{\bold{m}}
\newcommand{\bn}{\bold{n}}
\newcommand{\bo}{\bold{o}}
\newcommand{\bp}{\bold{p}}
\newcommand{\bq}{\bold{q}}
\newcommand{\br}{\bold{r}}
\newcommand{\bs}{\bold{s}}
\newcommand{\bt}{\bold{t}}
\newcommand{\bu}{\bold{u}}
\newcommand{\bv}{\bold{v}}
\newcommand{\bw}{\bold{w}}
\newcommand{\bx}{\bold{x}}
\newcommand{\by}{\bold{y}}
\newcommand{\bz}{\bold{z}}
\newcommand{\bA}{\bold{A}}
\newcommand{\bB}{\bold{B}}
\newcommand{\bC}{\bold{C}}
\newcommand{\bD}{\bold{D}}
\newcommand{\oD}{\overline{D}}
\newcommand{\bE}{\bold{E}}
\newcommand{\bF}{\bold{F}}
\newcommand{\bG}{\bold{G}}
\newcommand{\bH}{\bold{H}}
\newcommand{\bI}{\bold{I}}
\newcommand{\bJ}{\bold{J}}
\newcommand{\bK}{\bold{K}}
\newcommand{\bL}{\bold{L}}
\newcommand{\bM}{\bold{M}}
\newcommand{\bN}{\bold{N}}
\newcommand{\bO}{\bold{O}}
\newcommand{\bP}{\bold{P}}
\newcommand{\bQ}{\bold{Q}}
\newcommand{\bR}{\bold{R}}
\newcommand{\bS}{\bold{S}}
\newcommand{\bT}{\bold{T}}
\newcommand{\bU}{\bold{U}}
\newcommand{\bV}{\bold{V}}
\newcommand{\bW}{\bold{W}}
\newcommand{\bX}{\bold{X}}
\newcommand{\oX}{\overline{X}}
\newcommand{\bY}{\bold{Y}}
\newcommand{\bZ}{\bold{Z}}
\newcommand{\calS}{\mathcal{S}}
%%%%%%%%%%%%%%%%%%%%%%%%%%%%
%%%%%%%%%%%%%%%%%%%%%%%%%%%%
%%%%%%%%%%%%%%%%%%%%%%%%%%%%
% definitions for hyperref %
%%%%%%%%%%%%%%%%%%%%%%%%%%%%
%fix counters
%\usepackage{aliascnt}

%%%%%%%%%%%%%%%%%%%%%%%%%%%%
% Optimization problems    %
%%%%%%%%%%%%%%%%%%%%%%%%%%%%
\newcommand{\probmtflprimal}{\mathcal{E}_{\mbox{\tiny MTFL}}}
\newcommand{\probmtfldual}{\mathcal{C}_{\mbox{\tiny MTFL}}}
\newcommand{\probmtfldualeps}{\mathcal{C}_{\mbox{$\varepsilon$-\tiny {MTFL}}}}
\newcommand{\mtflmodprimal}{\mathcal{E}}
\newcommand{\mtflmoddual}{\mathcal{C}}
\newcommand{\mtflmoddualeps}{{\mathcal{C}_{\vareps}}}
\newcommand{\mtflmoddualmerge}{{\mathcal{S}_{\vareps}}}
%%%%%%%%%%%%%%%%%%%%%%%%%%%%
% Algorithms stuff         %
%%%%%%%%%%%%%%%%%%%%%%%%%%%%
%\algnewcommand\algorithmicinput{\textbf{input:}}
%\algnewcommand\Input{\item[\algorithmicinput]}
%\algnewcommand\algorithmicoutput{\textbf{output:}}
%\algnewcommand\Output{\item[\algorithmicoutput]}
%\algdef{SE}[DOWHILE]{Do}{DoWhile}{\algorithmicdo}[1]{\algorithmicwhile\ #1}
%%%%%%%%%%%%%%%%%%%%%%%%%%%%
% Matrix operators         %
%%%%%%%%%%%%%%%%%%%%%%%%%%%%
\newcommand{\transp}{\mathsf{T}}
\DeclareMathOperator*{\argmin}{arg\,min}
\DeclareMathOperator*{\argmax}{arg\,max}
\DeclareMathOperator*{\trace}{trace}
\DeclareMathOperator*{\range}{Ran}
\DeclareMathOperator*{\kernel}{Ker}
\DeclareMathOperator*{\diag}{Diag}
\DeclareMathOperator*{\Vecm}{Vec}
\DeclareMathOperator*{\Rank}{Rank}
\DeclareMathOperator*{\Row}{Row}
\DeclareMathOperator*{\Col}{Col}
\DeclareMathOperator*{\Op}{\textbf{op}}
\DeclareMathOperator*{\nnz}{\textbf{nnz}}
%%%%%%%%%%%%%%%%%%%%%%%%%%%%
% Norms                    %
%%%%%%%%%%%%%%%%%%%%%%%%%%%%
\newcommand{\norm}[2]{\left\Vert #1 \right\Vert_{#2}}
\newcommand{\normsmall}[2]{\Vert #1 \Vert_{#2}}
%%%%%%%%%%%%%%%%%%%%%%%%%%%%
% Statistic operators      %
%%%%%%%%%%%%%%%%%%%%%%%%%%%%
\newcommand{\probability}{\mathbb{P}}
\newcommand{\probdist}{Pr}
\DeclareMathOperator*{\expectedvalue}{\mathbb{E}}
\newcommand{\expectedvalueover}[1]{\expectedvalue\limits_{#1}}
\newcommand{\gaussdistr}{\mathcal{N}}
\newcommand{\uniformdistr}{\mathcal{U}}
\newcommand{\indvec}{J}
%%%%%%%%%%%%%%%%%%%%%%%%%%%%
% Algebraic Sets           %
%%%%%%%%%%%%%%%%%%%%%%%%%%%%
\newcommand{\realnumbers}{\mathbb{R}}
\newcommand{\realnumberssdp}{\mathbb{R}_{+}}
\newcommand{\realnumbersdp}{\mathbb{R}_{++}}
\newcommand{\naturalnumbers}{\mathbb{N}}
\newcommand{\symmetricsdp}{\boldsymbol{S}^{d}_{+}}
\newcommand{\symmetricdp}{\boldsymbol{S}^{d}_{++}}
\newcommand{\symmetric}{\boldsymbol{S}^{d}}
\newcommand{\orthogonal}{\boldsymbol{O}^{d}}
%%%%%%%%%%%%%%%%%%%%%%%%%%%%
% MDP related notation     %
%%%%%%%%%%%%%%%%%%%%%%%%%%%%
\newcommand{\statespace}{\mathcal{X}}
\newcommand{\statespacedisc}{S}
\newcommand{\actionspace}{\mathcal{A}}
\newcommand{\transitionspace}{\mathcal{P}}
\newcommand{\transitionkernel}{P}
\newcommand{\rewardspace}{\mathcal{R}}
\newcommand{\rewardkernel}{R}
\newcommand{\epochset}{\mathcal{T}}
\newcommand{\trajspace}{\mathcal{H}}
\newcommand{\funcspace}{\mathcal{F}}
\newcommand{\dataset}{\mathcal{D}}
\newcommand{\Qfuncspace}{\mathcal{Q}}
\newcommand{\borelsetspace}{\mathcal{B}}
\newcommand{\kernelspace}{\mathcal{K}}
%\newcommand{\muset}{\boldsymbol\mu}
\newcommand{\muset}{\mu_{\pi}}
%\newcommand{\rhoset}{\boldsymbol\rho}
\newcommand{\rhoset}{\rho_{\pi}}
\newcommand{\rewardaverage}{R}
\newcommand{\rewardsample}{r}
\newcommand{\rewardsamplevec}{\br}
%\newcommand{\optbellop}{\mathcal{T}^{*}}
\newcommand{\optbellop}{\mathcal{T}}
\newcommand{\T}{\mathcal{T}}
\newcommand{\D}{\mathcal{P}}
\renewcommand{\Re}{\mathbb{R}}
%%%%%%%%%%%%%%%%%%%%%%%%%%%%
%%%%%%%%%%%%%%%%%%%%%%%%%%%%
%Short version of notation %
%%%%%%%%%%%%%%%%%%%%%%%%%%%%
\newcommand{\wrt}{w.r.t. }
%\newcommand{\defeq}{\stackrel{\mathclap{\normalfont\mbox{\tiny def}}}{=}}
\newcommand{\maxund}[1]{\max\limits_{#1}}
\newcommand{\minund}[1]{\min\limits_{#1}}
\newcommand{\vareps}{\varepsilon}
\newcommand{\bigotime}{\mathcal{O}}
%%%%%%%%%%%%%%%%%%%

%%%%%%%%%%%%%%%%%%%

%\newtheorem{theorem}{Theorem}
%\newtheorem{lemma}{Lemma}
%\newtheorem{corollary}{Corollary}
%\newtheorem{proposition}{Proposition}
%%\newtheorem{definition}{Definition}
%\newtheorem{assumption}{Assumption}
%%\newtheorem{remark}{Remark}

\newcommand{\gl}{{\sc GL}\xspace}
\newcommand{\mtfl}{{\sc MTFL}\xspace}
\newcommand{\fqi}{{\sc F$Q$I}\xspace}
\newcommand{\lassofqi}{{\sc LASSO--F$Q$I}\xspace}
\newcommand{\glfqi}{{\sc GL--F$Q$I}\xspace}
\newcommand{\flfqi}{{\sc FL--F$Q$I}\xspace}
\newcommand{\lasso}{{\sc LASSO}\xspace}



%\newaliascnt{lemma}{theorem}
%\newtheorem{lemma}[lemma]{Lemma}
%\aliascntresetthe{lemma}
%\providecommand*{\lemmaautorefname}{Lemma}
%\newaliascnt{corollary}{theorem}
%\newtheorem{corollary}[corollary]{Corollary}
%\aliascntresetthe{corollary}
%\providecommand*{\corollaryautorefname}{Corollary}
%\newaliascnt{proposition}{theorem}
%\newtheorem{proposition}[proposition]{Proposition}
%\aliascntresetthe{proposition}
%\providecommand*{\propositionautorefname}{Proposition}
%\newaliascnt{definition}{theorem}
%\newtheorem{definition}[definition]{Definition}

%\aliascntresetthe{definition}
%\providecommand*{\definitionautorefname}{Definition}
%\newaliascnt{assumption}{theorem}
%\newtheorem{assumption}[assumption]{Assumption}
%\aliascntresetthe{assumption}
%\providecommand*{\assumptionautorefname}{Assumption}


%\newenvironment{remark}[1][Remark]{\begin{trivlist}
%\item[\hskip \labelsep {\bfseries #1}]}{\end{trivlist}}





\setlength{\parskip}{\baselineskip}%


\begin{document}


\section{Online SSL}

\begin{itemize}
	\item Complete \path{online_ssl_update_centroids} using the pseudocode \ref{alg:inc_k_centers}.
	\item Complete \path{online_ssl_compute_solution} following the
pseudocode \ref{alg:quant_shfs}
\end{itemize}



\begin{algorithm}[H]
	\begin{algorithmic}[1]
		{\small
			\STATE  \textbf{Input:} an unlabeled $x_t$, a list of centroids $C_{t - 1}$, a list of multiplicities $v_{t-1}$, taboo list $b$ containing the labeled centroids.
			%     {\bf Algorithm:} \\
			\IF{$(|C_{t - 1}| = k)$} 
				\STATE $c_1, c_2 \gets $ two closest centroids such that at least one of them is not in $b$.
				\STATE // Decide which centroid is $c_{\mathrm{rep}}$, that will represent both $c_1$ and $c_2$, and which centroid is $c_{\mathrm{add}}$, that will represent the new point $x_t$.
				\IF{$c_1$ in b}
					\STATE $c_{\mathrm{rep}} \gets c_1$
					\STATE $c_{\mathrm{add}} \gets c_2$
				\ELSIF{$c_2$ in b}
					\STATE $c_{\mathrm{rep}} \gets c_2$
					\STATE $c_{\mathrm{add}} \gets c_1$
				\ELSIF{$v_{t-1}(c_2) \leq  v_{t-1}(c_1)$}
					\STATE $c_{\mathrm{rep}} \gets c_1$
					\STATE $c_{\mathrm{add}} \gets c_2$ 
				\ELSE 
					\STATE $c_{\mathrm{rep}} \gets c_2$
					\STATE $c_{\mathrm{add}} \gets c_1$
				\ENDIF  
				\STATE $v_t \gets v_{t-1}$
				\STATE $v_t(c_{\mathrm{rep}}) \gets v_t(c_{\mathrm{rep}}) + v_t(c_{\mathrm{add}})$
				\STATE $c_{\mathrm{add}} \gets x_t$
				\STATE $v_t(c_{\mathrm{add}}) = 1$
			\ELSE
				\STATE  $C_t \gets C_{t - 1}.\mathrm{append}(x_t)$
				\STATE  $v_t \gets v_{t-1}.\mathrm{append}(1)$
			\ENDIF
		}
	\end{algorithmic}
	\caption{Incremental $k$-centers (simplified)}\label{alg:inc_k_centers}
\end{algorithm}

\begin{algorithm}[H]
	\begin{algorithmic}[1]
		\STATE  \textbf{Input:} $t$, a list of centroids $C_t$, a list of multiplicities $v_t$ and labels $y$.
		\STATE $V \gets \mathrm{diag}(v_t)$
		\STATE $[\widetilde{W}_q]_{ij} \gets$ weight between centroids $i$ and $j$. 
		\STATE Compute the Laplacian $L$ of the graph represented by $W_q = V\widetilde{W}_qV$
		\STATE  // Infer labels using hard-HFS.
		\STATE $\widehat{y_t} \gets \mathrm{hardHFS}(L, y)$
		\STATE // Remark: with the preceding construction of the centroids, $x_t$ is always present in the reduced graph and does not share the centroid with any other node.
	\end{algorithmic}
	\caption{Online HFS with Graph Quantization}\label{alg:quant_shfs}
\end{algorithm}


Some practical considerations: 
\begin{itemize}
	\item The labeled nodes are fundamentally different from unlabeled ones.
	Because of this, it is always a good idea to keep them separate, and never merge them in a centroid. In the implementation this is accomplished with a taboo list $b$ that keeps track of nodes that cannot be merged together.
	\item In streaming applications, it is not always possible to stop execution to partition the centroids,
	and it is often preferable to pay a small price at every step to keep execution smooth. In our case, the centroids are updated at every step.
	\item Whenever a new node arrives, and we have too many centroids, we choose the two closest centroids $c_{\mathrm{add}}$ and $c_{\mathrm{rep}}$.
	$c_{\mathrm{add}}$ will forget the old centroid and will point to the
	new sample that just arrived, and $c_{\mathrm{rep}}$
	will take care of representing all nodes that belonged to $c_{\mathrm{add}}$.
\end{itemize}



\bibliographystyle{plain}
\bibliography{td}


\end{document}
